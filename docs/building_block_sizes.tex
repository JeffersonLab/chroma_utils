\documentclass[12pt]{article}
\begin{document}

\begin{center}
\bf Sizes of Building Block Files\\
Dru Renner\\
June 20, 2007
\end{center}


The exact formula for the size of a building block file is given below, but
first, here is a more useful and accurate upper bound.
In this approximation, the size of the building block file depends on the number of link patterns,
$N_\mathrm{L}$, the maximum length of the link patterns, $l_{\mathrm{max}}$,
the number of momentum transfers, $N_{\mathrm{Q}}$, and
the number of time slices, $N_{\mathrm{T}}$, included in the file.
Here is the upper bound on the size, $\mathrm{S}$, of a building block file for version 3.
%
\begin{displaymath}
\mathrm{S} \le 104 + N_{\mathrm{L}} \left\{ 2(1+l_{\mathrm{max}}) + N_{\mathrm{Q}} 16 ( 6 + 8 N_{\mathrm{T}} ) \right\}
\end{displaymath}
%
To illustrate the formula, consider our current calculations with $N_{\mathrm{T}} = 64$ and
one momentum transfer per file, $N_{\mathrm{Q}} = 1$.  For the $\vec{p}=0$ sink we have
$N_{\mathrm{L}} = 65$ and for the $\vec{p}\not=0$ sink we have $N_{\mathrm{L}} = 457$.  The
corresponding upper bounds are 539214 and 3791376.  You can compare these with the exact
sizes of 539194 and 3791226 respectively.

The above result is probably the most useful formula, but to be precise here is the exact
formula.  You must determine the number of link patterns of length $l$, $N_{\mathrm{L}}^l$.
Then the exact formula is
%
\begin{displaymath}
\mathrm{S} = 104 + \sum_l N_{\mathrm{L}}^l \left\{ 2(1+l) + N_{\mathrm{Q}} 16 ( 6 + 8 N_{\mathrm{T}} ) \right\}
\end{displaymath}
%
You can easily check that the previous result is indeed a valid upper bound.

If you keep all link patterns of a fixed size, then there is a formula for the number
of link patterns, $N^l$, of length $l$.  For $l=0$ we have
$N^{{}_0}=1$ and $N^l=8\cdot 7^{l-1}$ for $l>0$.
The first few are given here, along with the cumulative count, for ease of reference.
%
\begin{displaymath}
\begin{array}{|c|c|c|}\hline
l & N^l & \sum_{m=0}^l N^m\\\hline
0 &    1 & 1    \\
1 &    8 & 9    \\
2 &   56 & 65   \\
3 &  392 & 457  \\
4 & 2744 & 3201 \\\hline
\end{array}
\end{displaymath}

I don't have a simple formula for the momentum transfers, but you can easily count.  Here
are the first few cases where $N$ is the number of equivalent $\vec{q}$ for the given case
and $C$ is the cumulative count.
%
\begin{displaymath}
\begin{array}{|c|c|c|c|}\hline
q^2 & \mathrm{ex.}~\vec{q} & N  & C   \\\hline
0   & (0,0,0)              & 1  & 1   \\
1   & (1,0,0)              & 6  & 7   \\
2   & (1,1,0)              & 12 & 19  \\
3   & (1,1,1)              & 8  & 27  \\
4   & (2,0,0)              & 6  & 33  \\
5   & (2,1,0)              & 24 & 57  \\
6   & (2,1,1)              & 24 & 81  \\
8   & (2,2,0)              & 12 & 93  \\
9   & (2,2,1)              & 24 & 117 \\
12  & (2,2,2)              &  8 & 124 \\\hline
\end{array}
\end{displaymath}


I should note that I have not actually seen a building block file from chroma with
$N_{\mathrm{Q}} \not= 1$, so I am guessing about the $N_{\mathrm{Q}}$ dependence.  This
note might need to be modified to correct for that.

\end{document}
